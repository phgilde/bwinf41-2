\documentclass[a4paper,10pt,ngerman]{scrartcl}
\usepackage{babel}
\usepackage[T1]{fontenc}
\usepackage[utf8x]{inputenc}
\usepackage[a4paper,margin=2.5cm,footskip=0.5cm]{geometry}

% Die nächsten drei Felder bitte anpassen:
\newcommand{\Aufgabe}{Aufgabe 1: \LaTeX-Dokument} % Aufgabennummer und Aufgabennamen angeben
\newcommand{\TeilnahmeId}{?????}                  % Teilnahme-ID angeben
\newcommand{\Name}{Vor- und Nachname}             % Name des Bearbeiter / der Bearbeiterin dieser Aufgabe angeben

% Kopf- und Fußzeilen
\usepackage{scrlayer-scrpage, lastpage}
\setkomafont{pageheadfoot}{\large\textrm}
\lohead{\Aufgabe}
\rohead{Teilnahme-ID: \TeilnahmeId}
\cfoot*{\thepage{}/\pageref{LastPage}}

% Position des Titels
\usepackage{titling}
\setlength{\droptitle}{-1.0cm}

% Für mathematische Befehle und Symbole
\usepackage{amsmath}
\usepackage{amssymb}

% Für Bilder
\usepackage{graphicx}

% Für Algorithmen
\usepackage{algpseudocode}

\usepackage{tabularx}
\usepackage{booktabs}

% Für Quelltext
\usepackage{listings}
\usepackage{color}
\definecolor{mygreen}{rgb}{0,0.6,0}
\definecolor{mygray}{rgb}{0.5,0.5,0.5}
\definecolor{mymauve}{rgb}{0.58,0,0.82}
\lstset{
  keywordstyle=\color{blue},commentstyle=\color{mygreen},
  stringstyle=\color{mymauve},rulecolor=\color{black},
  basicstyle=\footnotesize\ttfamily,numberstyle=\tiny\color{mygray},
  captionpos=b, % sets the caption-position to bottom
  keepspaces=true, % keeps spaces in text
  numbers=left, numbersep=5pt, showspaces=false,showstringspaces=true,
  showtabs=false, stepnumber=2, tabsize=2, title=\lstname
}
\lstdefinelanguage{JavaScript}{ % JavaScript ist als einzige Sprache noch nicht vordefiniert
  keywords={break, case, catch, continue, debugger, default, delete, do, else, finally, for, function, if, in, instanceof, new, return, switch, this, throw, try, typeof, var, void, while, with},
  morecomment=[l]{//},
  morecomment=[s]{/*}{*/},
  morestring=[b]',
  morestring=[b]",
  sensitive=true
}

% Diese beiden Pakete müssen zuletzt geladen werden
%\usepackage{hyperref} % Anklickbare Links im Dokument
\usepackage{cleveref}

% Daten für die Titelseite
\title{\textbf{\Huge\Aufgabe}}
\author{\LARGE Teilnahme-ID: \LARGE \TeilnahmeId \\\\
  \LARGE Bearbeiter/-in dieser Aufgabe: \\
  \LARGE \Name\\\\}
\date{\LARGE\today}

\begin{document}

\maketitle
\tableofcontents

\vspace{0.5cm}

\textbf{Anleitung:} Trage oben in den Zeilen 8 bis 10 die Aufgabennummer, die Teilnahme-ID und die/den Bearbeiterin/Bearbeiter dieser Aufgabe mit Vor- und Nachnamen ein.
Vergiss nicht, auch den Aufgabennamen anzupassen (statt "`\LaTeX-Dokument"')!

Dann kannst du dieses Dokument mit deiner \LaTeX-Umgebung übersetzen.

Die Texte, die hier bereits stehen, geben ein paar Hinweise zur Einsendung. Du
solltest sie aber in deiner Einsendung wieder entfernen!

\section{Lösungsidee}
\subsection{Notation}
Die Menge der möglichen Stapel der Höhe $n$ wird mit $\mathcal{P}_n$ bezeichnet.
Die Möglichen Pfannkuchen-Wende-und-Ess-Operationen für einem Stapel mit $n$ Pfannkuchen
wird mit $\mathcal{W}_n$ bezeichnet. Die Menge der möglichen Umkehroperationen für solch einen Stapel
wird mit $\mathcal{W}^{-1}_n$ bezeichnet. Wenn der Stapel $S \in \mathcal{P}_n$ durch die Operation
$w \in \mathcal{W}_n$ verändert wird, so wird der neue Stapel $S' = w S$ bezeichnet. Operationen
assoziieren nach rechts, d.h. $w_1 w_2 S = w_1 (w_2 S)$. Die Funktionen $A$ und $P$ werden aus
der Aufgabenstellung übernommen.
\subsection{PWUE-Zahl}
Die PWUE-Zahl kann rekursiv mit Hilfe der dynamischen Programmierung berechnet werden.
Dafür definieren wir die Funktion $K(n,a)=\{s \in \mathcal{P}_n \mid A(s) = a\}$, die die Menge aller
Stapel der Höhe $n$ enthält, die in mindestens $a$ Schritten sortiert werden können.
Die Funktion lässt sich rekursiv berechnen:
\begin{align*}
  K(n,a) &= \{ws, w \in \mathcal{W}^{-1}_{n-1}, s \in K(n-1,a-1)\} | \forall v \in \mathcal{W}_n: A(vws) \geq a-1\} \\
  &= \{ws, w \in \mathcal{W}^{-1}_{n-1}, s \in K(n-1,a-1)\} | \forall v \in \mathcal{W}_n: \exists b \geq a-1: A(vws) = b\} \\
  &= \{ws, w \in \mathcal{W}^{-1}_{n-1}, s \in K(n-1,a-1)\} | \forall v \in \mathcal{W}_n: \exists b \geq a-1: vws \in K(n-1,b)\} \\
\end{align*}
$K(n,a)$ enthält also alle Stapel, die durch eine Umkehroperation aus Stapeln der Höhe $n-1$ mit mindestens $a-1$ Sortieroperationen entstehen können
und für die keine andere Sortieroperation eine Stapel bildet, der in weniger als $a-1$ Schritten sortiert werden kann.
Nach dieser Definition würde $K(n, 1)$ allerdings auch die komplett sortierten Stapel enthalten, weshalb noch die Bedingung
$(a>1)\vee (s \notin K(n,0))$ ergänzt werden muss. Die Funktion $K(n,a)$ ist also definiert als
\begin{align*}
  K(n,a) &= \{ws, w \in \mathcal{W}^{-1}_{n-1}, s \in K(n-1,a-1)\} | \forall v \in \mathcal{W}_n: \exists b \geq a-1: vws \in K(n-1,b) \wedge ((a>1)\vee (s \notin K(n,0)))\} \\
\end{align*}
Dass diese Definition richtig ist, lässt sich überprüfen durch die Substitution
$A(S)=k, S \in \mathcal{P}_n \iff (\forall w \in \mathcal{W}_n: A(ws) \geq k-1)\wedge(\exists w \in \mathcal{W}_n: A(ws) = k-1)$.
Um jetzt die PWUE-Zahl zu berechnen, muss nur noch die Funktion $K(n,a)$ für alle $a$ berechnet werden und überprüft werden, ob sie Elemente enthält.
Da jeder Stapel $S \in \mathcal{P}_n$ in $\lceil \frac{n}{1.5}\rceil$ Schritten sortiert werden kann, reicht es aus, die Funktion $K(n,a)$ für alle $\lceil \frac{n}{1.5}\rceil$ zu berechnen.
Damit lässt sich auch $\exists b \geq a-1: vws \in K(n-1,b)$ durch $\exists \lceil \frac{n}{1.5}\rceil \geq b \geq a-1: vws \in K(n-1,b)$ ersetzen wodurch nicht unendlich viele Werte für $b$
ausprobiert werden müssen. Zuletzt muss noch ein Ende der Rekursion eingeführt werden, wir setzen
$K(n,0) = \{(1, \dots, n)\}$.
\section{Umsetzung}
Hier wird kurz erläutert, wie die Lösungsidee im Programm tatsächlich umgesetzt
wurde. Hier können auch Implementierungsdetails erwähnt werden.

\section{Beispiele}
Genügend Beispiele einbinden! Die Beispiele von der BwInf-Webseite sollten hier
diskutiert werden, aber auch eigene Beispiele sind sehr gut – besonders wenn
sie Spezialfälle abdecken. Aber bitte nicht 30 Seiten Programmausgabe hier
einfügen!

\section{Quellcode}
Unwichtige Teile des Programms sollen hier nicht abgedruckt werden. Dieser Teil
sollte nicht mehr als 2–3 Seiten umfassen, maximal 10.

\end{document}