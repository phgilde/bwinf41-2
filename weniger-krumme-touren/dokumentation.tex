\documentclass[a4paper,10pt,ngerman]{scrartcl}
\usepackage{babel}
\usepackage[T1]{fontenc}
\usepackage[utf8x]{inputenc}
\usepackage[a4paper,margin=2.5cm,footskip=0.5cm]{geometry}

% Die nächsten drei Felder bitte anpassen:
\newcommand{\Aufgabe}{Aufgabe 1: \LaTeX-Dokument} % Aufgabennummer und Aufgabennamen angeben
\newcommand{\TeilnahmeId}{?????}                  % Teilnahme-ID angeben
\newcommand{\Name}{Vor- und Nachname}             % Name des Bearbeiter / der Bearbeiterin dieser Aufgabe angeben

% Kopf- und Fußzeilen
\usepackage{scrlayer-scrpage, lastpage}
\setkomafont{pageheadfoot}{\large\textrm}
\lohead{\Aufgabe}
\rohead{Teilnahme-ID: \TeilnahmeId}
\cfoot*{\thepage{}/\pageref{LastPage}}

% Position des Titels
\usepackage{titling}
\setlength{\droptitle}{-1.0cm}

% Für mathematische Befehle und Symbole
\usepackage{amsmath}
\usepackage{amssymb}

% Für Bilder
\usepackage{graphicx}

% Für Algorithmen
\usepackage{algpseudocode}
\usepackage{amsthm}
\usepackage{tabularx}
\usepackage{booktabs}

% Für Quelltext
\usepackage{listings}
\usepackage{color}
\definecolor{mygreen}{rgb}{0,0.6,0}
\definecolor{mygray}{rgb}{0.5,0.5,0.5}
\definecolor{mymauve}{rgb}{0.58,0,0.82}
\lstset{
  keywordstyle=\color{blue},commentstyle=\color{mygreen},
  stringstyle=\color{mymauve},rulecolor=\color{black},
  basicstyle=\footnotesize\ttfamily,numberstyle=\tiny\color{mygray},
  captionpos=b, % sets the caption-position to bottom
  keepspaces=true, % keeps spaces in text
  numbers=left, numbersep=5pt, showspaces=false,showstringspaces=true,
  showtabs=false, stepnumber=2, tabsize=2, title=\lstname
}
\lstdefinelanguage{JavaScript}{ % JavaScript ist als einzige Sprache noch nicht vordefiniert
  keywords={break, case, catch, continue, debugger, default, delete, do, else, finally, for, function, if, in, instanceof, new, return, switch, this, throw, try, typeof, var, void, while, with},
  morecomment=[l]{//},
  morecomment=[s]{/*}{*/},
  morestring=[b]',
  morestring=[b]",
  sensitive=true
}

% Diese beiden Pakete müssen zuletzt geladen werden
%\usepackage{hyperref} % Anklickbare Links im Dokument
\usepackage{cleveref}

% Daten für die Titelseite
\title{\textbf{\Huge\Aufgabe}}
\author{\LARGE Teilnahme-ID: \LARGE \TeilnahmeId \\\\
  \LARGE Bearbeiter/-in dieser Aufgabe: \\
  \LARGE \Name\\\\}
\date{\LARGE\today}
\newtheorem{theorem}{Satz}
\newtheorem{lemma}[theorem]{Lemma}
\begin{document}

\maketitle
\tableofcontents

\vspace{0.5cm}

\textbf{Anleitung:} Trage oben in den Zeilen 8 bis 10 die Aufgabennummer, die Teilnahme-ID und die/den Bearbeiterin/Bearbeiter dieser Aufgabe mit Vor- und Nachnamen ein.
Vergiss nicht, auch den Aufgabennamen anzupassen (statt "`\LaTeX-Dokument"')!

Dann kannst du dieses Dokument mit deiner \LaTeX-Umgebung übersetzen.

Die Texte, die hier bereits stehen, geben ein paar Hinweise zur Einsendung. Du
solltest sie aber in deiner Einsendung wieder entfernen!

\section{Lösungsidee}
\subsection{Einleitung}
Es ist $\mathcal{NP}$-schwer, das weniger krumme Touren-Problem (WKT) optimal zu lösen. Um das zu zeigen, wird
eine Reduktion vom eulerschen Pfad-Problem des Handlungsreisenden (E-PTSP) skizziert,
 welches bekanntermaßen $\mathcal{NP}$-schwer ist.
 E-PTSP lautet folgendermaßen: Sei $P \subset \mathbb{R}^2$ eine endliche Menge. Dann wird eine Reihenfolge von P gesucht,
 bei der die Strecke zwischen aufeinanderfolgenden Punkten minimal ist. \\
 E-PTSP kann nun auf WKT reduziert werden, in dem jeder der Punkte $P$ durch eine Struktur $S$ mit
 sehr kleinem $d$ ersetzt wird. Wie das Diagramm zeigt, kann Struktur S aus beliebigen Richtungen angeflogen werden.
 Wenn die Struktur sehr klein ist, verhält sie sich wie ein Punkt im E-PTSP. Die Reihenfolge, in welcher die Strukturen optimal angeflogen werden,
 ist demnach auch die optimale Reihenfolge der Punkte $P$ für E-PTSP. \\\\
 Unter der Annahme $\mathcal{P} \neq \mathcal{NP}$ gibt es deshalb keinen Algorithmus, der WKT in polynomieller Zeit optimal lößt.
 Stattdessen stelle ich einen Algorithmus vor, der eine optimale Lösung in polynomieller Zeit annähert,
 sowie einen Lösungsansatz, der das Problem in exponentieller Zeit optimal lößt.
 \subsection{Annäherung}
 Das Problem wird mit Hilfe des simulated Annealing gelößt.
 \subsection{Optimale Lösung}
 Zur optimalen Lösung von WKT wird dieses als Integer-Programming-Problem formuliert. Integer Programming ist
 $\mathcal{NP}$-schwer, weshalb dafür nur Algorithmen exponentieller Laufzeit bekannt sind. \\
Das verwandte Problem des Handlungsreisenden kann durch die Formulierung von Miller-Tucker-Zenlin als Integer Programming
formuliert werden. 
 \section{Umsetzung}
Hier wird kurz erläutert, wie die Lösungsidee im Programm tatsächlich umgesetzt
wurde. Hier können auch Implementierungsdetails erwähnt werden.

\section{Beispiele}
Genügend Beispiele einbinden! Die Beispiele von der BwInf-Webseite sollten hier
diskutiert werden, aber auch eigene Beispiele sind sehr gut – besonders wenn
sie Spezialfälle abdecken. Aber bitte nicht 30 Seiten Programmausgabe hier
einfügen!

\section{Quellcode}
\end{document}